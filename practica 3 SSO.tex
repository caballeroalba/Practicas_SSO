\documentclass[a4paper]{article}
\usepackage[utf8]{inputenc}
\title{Práctica 3 }
\author{José María Caballero Alba}

\begin{document}

\maketitle

\newpage

\tableofcontents

\newpage

\section{Herramientas de Kali}
\subsection{a) Para qué sirve cada grupo de herramientas }

Tenemos varios grupos de herramientas, vamos listando por los siguientes:

\begin{enumerate}

\item Information gathering

Herramientas para la adquisición de información en sistemas, tanto en red (equipos externos) como el propio (información del equipo actual. 

Podemos ver entre ellas algunas como wireshark (equipos remotos y local) o nmap


\item VULNERABILITY ANALYSIS

Se trata de un grupo de herramientas para el análisis de vulnerabilidades en sistemas, algunas destacables son sqlninja o powerfuzzer

\item WIRELESS ATTACKS

Herramientas para el ataque a redes wifi, algunas destacabales son Aircrack-ng para ataque de a redes wpa, web y obtener la contraseña de red o pixieWPS para ataques off-line a redes con WPS activado


\item WEB APPLICATIONS

Herramientas para el ataque a aplicaciones web, algunas destacables son WebSploit o BurpSuite

\item EXPLOITATION TOOLS

Herramientas de explotación a varios objetivos distintos, algunas destacables son: Backdoor Factory o jboss-autopwn para aplicaciones web java

\item FORENSICS TOOLS

Herramientas para el análisis forense, algunas destacables son: Galleta para las cookies y iPhone Backup Analyzer para extraer información de datos de smartphones Iphone.

\item STRESS TESTING

Herramientas dedicadas al uso de aplicar test de stress al ordenador, algunas son: Termineter o SlowHTTPTest


\item SNIFFING \& SPOOFING

Herramientas para el esnifado de paquetes y envenenamiento de estos, algunas famosas son: sslstrip y responder

\item PASSWORD ATTACKS

Herramientas para el ataque a ficheros de pasword, algunas son John the Ripper y THC-Hydra




\item REVERSE ENGINEERING

Herramientas dedicadas a la ingeniería inversa, algunas son: Valgrind y apktool

\item HARDWARE HACKING

Herramientas dedicadas a la manipulación del hardware, algunas son : dex2jar y Sakis3G

\item REPORTING TOOLS

Herramientas dedicadas al reporte de datos sobre exámenes de seguridad a sistemas, algunas son: CaseFile y MagicTree


\end{enumerate}




\subsection{b) Comentar una herramienta, de vuestra elección, por grupo}


\begin{enumerate}

\item INFORMATION GATHERING

Herramienta: Nmap

Nmap es un escaner de puertos muy conocido en el ámbito de seguridad, permite hacer distintos escaneos de puertos en modos activo y pasivo, reconocimiento de sistemas operativos, versión de servidor usado (en servicios http, ftp, dns, etc) 

\item VULNERABILITY ANALYSIS

Herramienta: sqlninja

sqlninja es una herramienta para la inyección sql en bases de datos de Microsoft en aplicaciones web 

\item WIRELESS ATTACKS

Herramienta: aircrack


Aircrack es una conocida herramienta para la penetración en redes wifi, enfocadas especialmente en seguridad wpa y web, en la cual, en esta ultima tiene especial facilidad para poder penetrar en ellas. 

Se componen de varias herramientas como airodump-ng, airmon-ng, etc.

\item WEB APPLICATIONS

Herramienta BurpSuite

BurpSuite es una plataforma para perpetrar test de seguridad en aplicaciones web. Se compone de varias herramientas que trabajan conjuntamente para realizar los tests, desde el mapeo inicial y el análisis de la aplicación a atacar hasta buscar exploits y errores de seguridad. 


\item EXPLOITATION TOOLS

Herramienta jboss-autopwn

jboss-autopwn despliega un jsp para obtener una shell interactiva en el servidor jboss. Una vez desplegada permite al atacante ejecutar comandos mediante una sesión interactiva. 


\item FORENSICS TOOLS

Herramienta Iphone backup analyzer

Iphone backup analyzer permite visualizar el contenido de una copia de seguridad de un iphone creada por itunes.


\item STRESS TESTING

Herramienta Termineter

Termineter permite a cualquier persona conectarse a un contador inteligente de luz con la posibilidad de poder cambiar su software y así poder cambiar el consumo real que se esta utilizando por otro menor (o mayor).


\item SNIFFING \& SPOOFING


Herramienta sslstrip

sslstrip es una herramienta escrita en python que en conjunto con iptables y un buen ataque mitm permite al atacante evitar la victima se conecte mediante ssl emitiendo esta herramienta sus propios certificados de manera transparente a la victima. De esta manera podemos robar datos planos que irían cifrados.


\item PASSWORD ATTACKS

Herramienta John the Ripper

John the Ripper es una vieja conocida en el mundo del crackeo de password.

John the Ripper es una herramienta para conseguir contraseñas de ficheros de datos de usuario, como handshake, ficheros shadow, etc. 


\item REVERSE ENGINEERING

Herramienta apktool

aptktool es una herramienta para aplicar ingeniería inversa a los binarios cerrados de android. Puede descodificar elementos y volver a codificarlo aplicándole modificaciones varias.   



\item HARDWARE HACKING

Herramienta Sakis3G

Sakis3g es un script para crear una conexión a internet usando modems 3g. Es especialmente usada con dispositivos como las rasbperry pi dada su movilidad.


\item REPORTING TOOLS

Herramienta CaseFile

CaseFile es una aplicación de inteligencia visual que se puede utilizar para determinar las relaciones y enlaces del mundo real entre diferentes tipos de  de información.



\end{enumerate}

\section{Ejercicio 2  Comprobando la fortaleza de claves con jhon the ripper}

\subsection{ a) Indicar que tipos de ataques de claves permite utilizar }

Los únicos ataques que puede realizar a claves son por fuerza bruta y de diccionario. 


\subsection{ b) Utilizar la herramienta para comprobar la fortaleza de dos claves en nuestro sistema.}

¿Cuanto cuesta romper estas claves?

Se ha estado dejando el ordenador con jhon the ripper funcionando pero no se ha podido estimar cuando cuesta romper las claves puesto que la temperatura subía demasiado y el ordenador terminaba por apagarse.

Se ha usado ataque por fuerza bruta y por diccionario generado.

¿Cómo has generado estos diccionarios?
 
Usando crunch, exactamente de esta manera cruch 1 9 0123abcABC. No se ha usado otro juego de caracteres por posibles problemas de espacio de memoria y disco.



\end{document}